%  article.tex (Version 3.3, released 19 January 2008)
%  Article to demonstrate format for SPIE Proceedings
%  Special instructions are included in this file after the
%  symbol %>>>>
%  Numerous commands are commented out, but included to show how
%  to effect various options, e.g., to print page numbers, etc.
%  This LaTeX source file is composed for LaTeX2e.

%  The following commands have been added in the SPIE class 
%  file (spie.cls) and will not be understood in other classes:
%  \supit{}, \authorinfo{}, \skiplinehalf, \keywords{}
%  The bibliography style file is called spiebib.bst, 
%  which replaces the standard style unstr.bst.  

\documentclass[]{spie}  %>>> use for US letter paper
%%\documentclass[a4paper]{spie}  %>>> use this instead for A4 paper
%%\documentclass[nocompress]{spie}  %>>> to avoid compression of citations
%% \addtolength{\voffset}{9mm}   %>>> moves text field down
%% \renewcommand{\baselinestretch}{1.65}   %>>> 1.65 for double spacing, 1.25 for 1.5 spacing 
%  The following command loads a graphics package to include images 
%  in the document. It may be necessary to specify a DVI driver option,
%  e.g., [dvips], but that may be inappropriate for some LaTeX 
%  installations. 
\usepackage[]{graphicx}
\usepackage{caption} 
\usepackage{subcaption} 
%\usepackage{mathtools}

\title{CT Thermometry with Image Registration and Filtering} 

%>>>> The author is responsible for formatting the 
%  author list and their institutions.  Use  \skiplinehalf 
%  to separate author list from addresses and between each address.
%  The correspondence between each author and his/her address
%  can be indicated with a superscript in italics, 
%  which is easily obtained with \supit{}.

\author{Zachary DeStefano\supit{a}, Jack Yao\supit{a}, Nadine Abi-Jaoudeh\supit{a}, Ming Li\supit{a}
\skiplinehalf
\supit{a}National Institutes of Health, Clinical Center\\Bethesda, MD 20814
}

%>>>> Further information about the authors, other than their 
%  institution and addresses, should be included as a footnote, 
%  which is facilitated by the \authorinfo{} command.

\authorinfo{Author Emails: zdestefa@uci.edu}
%%>>>> when using amstex, you need to use @@ instead of @
 

%%%%%%%%%%%%%%%%%%%%%%%%%%%%%%%%%%%%%%%%%%%%%%%%%%%%%%%%%%%%% 
%>>>> uncomment following for page numbers
% \pagestyle{plain}    
%>>>> uncomment following to start page numbering at 301 
%\setcounter{page}{301} 
 
  \begin{document} 
  \maketitle 

%%%%%%%%%%%%%%%%%%%%%%%%%%%%%%%%%%%%%%%%%%%%%%%%%%%%%%%%%%%%% 
\begin{abstract}
Monitoring temperature during a cone-beam CT (CBCT) guided ablation procedure is important for prevention of over-treatment and under-treatment. In order to accomplish ideal temperature monitoring, a thermometry map must be generated.  Li et al explored this possibility using CBCT scans of a pig shoulder undergoing ablation. We are extending this work by using CBCT scans of real patients. Because we have real patients, we register the scans using NiftyReg **INSERT REF** before comparing them. Due to the noise in the image and remaining registration errors, we also employ a more robust change metric than simple image difference. 
\end{abstract}

%>>>> Include a list of keywords after the abstract 

\keywords{Cone-beam CT, CBCT, Thermometry, Radiofrequency Ablation, Time Delay Analysis}

\section{Introduction}
\label{sec:intro}  % \label{} allows reference to this section

Li et al explored the possibility of generating a thermal map from CBCT scans taken at different time points. We decided to extend this work and furthur explore thermometry map generation. We applied the methods to real patients. Because we have real patients, there is movement and deformation of the organs between CBCT scans. We thus decided to keep the baseline scan static and register the subsequent scans to the baseline one using **NIFTY**. After registration, there is still noise, needle artifacts, and small registration errors. We thus needed a change metric that was robust to these issues. We decided to compute the RMSE of the window around a given pixel. We treat the windows as a signal and find the ideal spatial delay between them and use the subsequent RMSE as the amount of change that occurred. 

The change metric we employ is computationally intensive. We thus only compute it around the Region of Interest (ROI) where the ablation occured. Because the needle has distinct Hounsfeld Unit values from tissue, we were able to isolate the needle and find the tip of it. The region around the tip was then our ROI. 

We used a data set of CBCT scans taken during 13 ablation procedures performed between September 2013 and June 2015. For each ablation procedure, there were between 1 and 4 ablations done. With each ablation, a baseline scan was taken followed by 1-4 comparison scans at different time points. Each of these time points had the ablation needle at different temperatures. We generated a thermal map at each of these time points in order to estimate the amount of tissue affected. 

\section{Related Work}

CT Thermometry has been explored since the 1970's with some success **CT OVERVIEW PAPER**. The attempts mentioned only focused on taking individual CT scan values at different points and doing a linear regression. Being that the CT scans have quite a lot of noise and artifacts, we decided to attempt to use a metric that accounted for it. Additionally, those attempts did not see what a full thermal map would look like.

Image Registration between 3D Medical Scans has been explored **NIFTY REFERENCES**. There is now an open source tool to accomplish this. We incorporated that tool in our work.  

A more robust change metric was proposed in **NADINE'S PAPER**. While this change metric would be good for classification, the formula makes it less than ideal for regression. We thus needed to figure out a change metric adapted for regression.

A change metric used in the pipeline is the difference between the convolutional average filtered image. Convolution is a common technique used in image processing. The averaging filter is a common method to reduce noise. We thus use it to give us denoised difference information. Unfortunately, it does not account for potential registration errors, so we needed something that adjusted to it. 

Our change metric ends up being RMSE between windows around a given pixel, but with an offset. Time delay analysis is a common field of study in signal processing. It is used to find the delay between signals. This is done by finding the cross correlation with different delays and seeing where it's maximized. The offset we use ends up being the same value. 

\section{Pipeline}

Here is a summary of the pipeline performed for each comparison scan to generate a thermal map:
\begin{enumerate}
\item Register the comparison scan to the baseline scan
\item Filter both images and calculate the difference
\item Use Connected Component and PCA to detect the Needles
\item Use the result to find the Region of Interest (ROI)
\item Calculate the Sliding Window RMSE value for the ROI
\item Correlate values from (5) in the temperature zones with the temperature data
\item Use regression from (6) to generate thermal map for entire ROI
\end{enumerate}


\subsection{Image Registration} 

Each comparison scan was registered to the baseline scan. Affine **INSERT REF** and Deformable Registration **INSERT REF** were performed using NiftyReg. Figure \ref{beforeAndAfterReg} shows an example of image slices before and after registration. 

\begin{figure} 
\centering 
\begin{subfigure}[t]{0.45\textwidth} 
\includegraphics[width=\textwidth]{unregisteredSlice.png}
\caption{Before Registration} 
\label{unregSlice} 
\end{subfigure} 
\begin{subfigure}[t]{0.45\textwidth} 
\includegraphics[width=\textwidth]{registeredSlice.png} 
\caption{After Registration} \label{regSlice} 
\end{subfigure} 
\caption{Baseline (green channel) and Comparison (red channel) Image superimposed in one image before and after registration}
\label{beforeAndAfterReg} 

\end{figure}

\subsection{Needle Detection}

The HU value for the needle and thermocouple were significantly higher than the values for normal tissue. I was thus able to easily find all the points in the 3D scan with the needle. 

**MENTION CONNECTED COMPONENT ANALYSIS DONE**

I then did PCA on these points in order find the direction of the needle. I was also able to obtain the endpoints of the needle by finding the min and max point in the principle PCA direction. I chose the end point closest to the center of the image as the probable end point. 

\subsection{Change Detection}

These images have a low signal to noise ratio as well as beam hardening artifacts. Because of this, a simple difference image is quite noisy. We decided to use an averaging filter to obtain the Region of Interest (ROI) that contained the ablation zone. We then calculate a Sliding Window RMSE value for each pixel in the ROI. 

The panel to the left shows different ways of comparing the ROI, including:
Raw Subtraction (top)
Average Filtered Image difference (middle)
Sliding Window method difference (bottom)

Raw Subtraction
Let A be baseline image 
Let B be comparison image 
Each pixel (i,j) in result image is as follows: A(i,j)-B(i,j)

Average Filtered Image difference
Let C be baseline image after applying an averaging filter
Let D be comparison image after applying an averaging filter
Each pixel (i,j) in result image is as follows: C(i,j)-D(i,j)

Sliding Window RMSE Method
For each pixel (i,j) in result image: 
Let U be neighborhood around (i,j) in the baseline image
Let V be neighborhood around (i,j) in the comparison image
Calculate the following:
\[
\min_{r,c} \sum_{l=0}^{m-1} \sum_{k=0}^{n-1} {(U_{k+r,l+c}-V_{k,l})^2}
\]
where it holds that
\[
U_{k+n,l+m} = U_{k,l}
\]


\subsection{Regression}

We then took the sliding window RMSE value and calibrated it using the measured temperature change at the needle through a regression model. We used the model to calculate a thermal map in the ROI. This thermal map was used to obtain an approximate mean temperature around the needles and thermocouples.

**INSERT POSTER IMAGES**


\section{Results}

\subsection{Finding Error Rates}

We decided to take the thermal maps generated from regression and see the average temperature in the zones that we were testing. We then did an RMSE of those temperatures with the measured temperatures. We dividied this RMSE by the temperature range in order to obtain an error rate. Here are the results:

**INSERT RESULT TABLE**

\subsection{Approximating Ablation Zone Area}

For the ablation zones, we made a graph of radius of region versus average temperature in that region in the thermal map. This graph can one day be used to approximate the radius of an ablation zone. Here it is for the various patients in our study. 

**INSERT GRAPH OF AREA VS AVERAGE TEMPERATURE**


\subsection{3D Visualization of Ablation Zone}

By taking the 3D Thermal Map, thresholding the values, and then using a surface generation algorithm **REFERENCE ITK-SNAP HERE**, we were able to create a 3D diagram of the ablation zone. We overlayed the 3D diagram of the needle and we were able to obtain a convincing visual. 

**SHOW VISUALS OF NEEDLE AND ABLATION ZONES**

\section{Conclusions and Future Work}

As can be observed, for some of the patients, our temperatures after regression were quite accurate while for others, the error was higher. The higher error was the result of a high residual value in the linear regression. This was likely caused by image noise, registration error, or user error when selecting the ROI. 

For the patients where the error was low, the methods we employed have the potential to provide useful thermal maps during ablation procedures. These thermal maps can then be used to approximate the size of the ablation zone. Additionally, there is the potential to generate a 3D visual representation of the ablation zone. 

The next step is to automate ROI selection using the needle properties. Additionally, we hope to use a data set where multiple imaging modalities were employed. That way we can test our thermal map against one generated by a modality that is known to generate accurate thermal maps. 



%\bibliography{report}   %>>>> bibliography data in report.bib
%\bibliographystyle{spiebib}   %>>>> makes bibtex use spiebib.bst

\end{document} 
