%  article.tex (Version 3.3, released 19 January 2008)
%  Article to demonstrate format for SPIE Proceedings
%  Special instructions are included in this file after the
%  symbol %>>>>
%  Numerous commands are commented out, but included to show how
%  to effect various options, e.g., to print page numbers, etc.
%  This LaTeX source file is composed for LaTeX2e.

%  The following commands have been added in the SPIE class 
%  file (spie.cls) and will not be understood in other classes:
%  \supit{}, \authorinfo{}, \skiplinehalf, \keywords{}
%  The bibliography style file is called spiebib.bst, 
%  which replaces the standard style unstr.bst.  

\documentclass[]{spie}  %>>> use for US letter paper
%%\documentclass[a4paper]{spie}  %>>> use this instead for A4 paper
%%\documentclass[nocompress]{spie}  %>>> to avoid compression of citations
%% \addtolength{\voffset}{9mm}   %>>> moves text field down
%% \renewcommand{\baselinestretch}{1.65}   %>>> 1.65 for double spacing, 1.25 for 1.5 spacing 
%  The following command loads a graphics package to include images 
%  in the document. It may be necessary to specify a DVI driver option,
%  e.g., [dvips], but that may be inappropriate for some LaTeX 
%  installations. 
\usepackage[]{graphicx}

\title{CT Thermometry with Image Registration and Filtering} 

%>>>> The author is responsible for formatting the 
%  author list and their institutions.  Use  \skiplinehalf 
%  to separate author list from addresses and between each address.
%  The correspondence between each author and his/her address
%  can be indicated with a superscript in italics, 
%  which is easily obtained with \supit{}.

\author{Zachary DeStefano\supit{a}, Jack Yao\supit{a}, Nadine Abi-Jaoudeh\supit{a}, Ming Li\supit{a}
\skiplinehalf
\supit{a}National Institute of Health, Clinical Center\\Bethesda, MD 20814
}

%>>>> Further information about the authors, other than their 
%  institution and addresses, should be included as a footnote, 
%  which is facilitated by the \authorinfo{} command.

\authorinfo{Author Emails: zdestefa@uci.edu}
%%>>>> when using amstex, you need to use @@ instead of @
 

%%%%%%%%%%%%%%%%%%%%%%%%%%%%%%%%%%%%%%%%%%%%%%%%%%%%%%%%%%%%% 
%>>>> uncomment following for page numbers
% \pagestyle{plain}    
%>>>> uncomment following to start page numbering at 301 
%\setcounter{page}{301} 
 
  \begin{document} 
  \maketitle 

%%%%%%%%%%%%%%%%%%%%%%%%%%%%%%%%%%%%%%%%%%%%%%%%%%%%%%%%%%%%% 
\begin{abstract}
The possibility of Cone-beam CT Thermometry was explored previously in **NADINE'S PAPER REF**. We seek to extend this work but applying it to real patients. This requires an additional step of doing image registration between cone-beam CT scans. Additionally, we vary the change metric used.  
\end{abstract}

%>>>> Include a list of keywords after the abstract 

\keywords{Cone-beam CT, Thermometry, Radiofrequency Ablation}

\section{Introduction}
\label{sec:intro}  % \label{} allows reference to this section

Monitoring temperature during ablation procedures is important for prevention of overtreatment and undertreatment. In order to accomplish ideal temperature monitoring, a thermometry map must be generated.  In particular, this must be possible for Cone-beam CT (CBCT) scans. This possibility was explored with CBCT scans of a pig shoulder phantom being ablated [1] . We are extending this work by using CBCT scans of real patients. Additionally, we are employing various image refinement techniques to improve the thermometry map. 

We used a data set of CBCT scans taken during 13 ablation procedures performed between September 2013 and June 2015. For each ablation procedure, there were between 1 and 4 ablations done. With each ablation, a baseline scan was taken followed by 1-4 comparison scans at different time points. Each of these time points had the ablation needle at different temperatures. We thus wanted to generate a thermal map at each of these time points in order to know how much tissue has been affected. 

For each comparison scan, the following was done to generate a thermal map:
\begin{enumerate}
\item Register the comparison scan to the baseline scan
\item Filter both images and calculate the difference
\item Use Connected Component and PCA to detect the Needles
\item Use the result to find the Region of Interest (ROI)
\item Calculate the Sliding Window RMSE value for the ROI
\item Correlate values from (5) in the temperature zones with the temperature data
\item Use regression from (6) to generate thermal map for entire ROI
\end{enumerate}

\section{Related Work}

CT Thermometry has been explored since the 1970's with some success **CT OVERVIEW PAPER**. The attempts mentioned only focused on taking individual CT scan values at different points and doing a linear regression. Being that the CT scans have quite a lot of noise and artifacts, we decided to attempt to use a metric that accounted for it. Additionally, those attempts did not see what a full thermal map would look like.

Image Registration between 3D Medical Scans has been explored **NIFTY REFERENCES**. There is now an open source tool to accomplish this. We incorporated that tool in our work.  

A more robust change metric was proposed in **NADINE'S PAPER**. While this change metric would be good for classification, the formula makes it less than ideal for regression. We thus needed to figure out a change metric adapted for regression.

A change metric used in the pipeline is the difference between the convolutional average filtered image. Convolution is a common technique used in image processing. The averaging filter is a common method to reduce noise. We thus use it to give us denoised difference information. Unfortunately, it does not account for potential registration errors, so we needed something that adjusted to it. 

Our change metric ends up being RMSE between windows around a given pixel, but with an offset. Time delay analysis is a common field of study in signal processing. It is used to find the delay between signals. This is done by finding the cross correlation with different delays and seeing where it's maximized. The offset we use ends up being the same value. 

\section{Pipeline}

\subsection{Image Registration} 

Each comparison scan was registered to the baseline scan. Affine and Deformable Registration were performed using NiftyReg [2].

\subsection{Needle Detection}

The HU value for the needle and thermocouple were significantly higher than the values for normal tissue. I was thus able to easily find all the points in the 3D scan with the needle. I then did PCA on these points in order find the direction of the needle. I was also able to obtain the endpoints of the needle by finding the min and max point in the principle PCA direction. I chose the end point closest to the center of the image as the probable end point. 

\subsection{Change Detection}

These images have a low signal to noise ratio as well as beam hardening artifacts. Because of this, a simple difference image is quite noisy. We decided to use an averaging filter to obtain the Region of Interest (ROI) that contained the ablation zone. We then calculate a Sliding Window RMSE value for each pixel in the ROI. 

The panel to the left shows different ways of comparing the ROI, including:
Raw Subtraction (top)
Average Filtered Image difference (middle)
Sliding Window method difference (bottom)

Raw Subtraction
Let A be baseline image 
Let B be comparison image 
Each pixel (i,j) in result image is as follows: A(i,j)-B(i,j)

Average Filtered Image difference
Let C be baseline image after applying an averaging filter
Let D be comparison image after applying an averaging filter
Each pixel (i,j) in result image is as follows: C(i,j)-D(i,j)

Sliding Window RMSE Method
For each pixel (i,j) in result image: 
Let U be neighborhood around (i,j) in the baseline image
Let V be neighborhood around (i,j) in the comparison image
Calculate the following:


\subsection{Regression}

\section{Results}

\section{Conclusions and Future Work}




%\bibliography{report}   %>>>> bibliography data in report.bib
%\bibliographystyle{spiebib}   %>>>> makes bibtex use spiebib.bst

\end{document} 
