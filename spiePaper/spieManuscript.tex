%  article.tex (Version 3.3, released 19 January 2008)
%  Article to demonstrate format for SPIE Proceedings
%  Special instructions are included in this file after the
%  symbol %>>>>
%  Numerous commands are commented out, but included to show how
%  to effect various options, e.g., to print page numbers, etc.
%  This LaTeX source file is composed for LaTeX2e.

%  The following commands have been added in the SPIE class 
%  file (spie.cls) and will not be understood in other classes:
%  \supit{}, \authorinfo{}, \skiplinehalf, \keywords{}
%  The bibliography style file is called spiebib.bst, 
%  which replaces the standard style unstr.bst.  

\documentclass[]{spie}  %>>> use for US letter paper
%%\documentclass[a4paper]{spie}  %>>> use this instead for A4 paper
%%\documentclass[nocompress]{spie}  %>>> to avoid compression of citations
%% \addtolength{\voffset}{9mm}   %>>> moves text field down
%% \renewcommand{\baselinestretch}{1.65}   %>>> 1.65 for double spacing, 1.25 for 1.5 spacing 
%  The following command loads a graphics package to include images 
%  in the document. It may be necessary to specify a DVI driver option,
%  e.g., [dvips], but that may be inappropriate for some LaTeX 
%  installations. 
\usepackage[]{graphicx}
\usepackage{caption} 
\usepackage{subcaption} 
%\usepackage{mathtools}

\title{CT Thermometry with Image Registration and Filtering} 

%>>>> The author is responsible for formatting the 
%  author list and their institutions.  Use  \skiplinehalf 
%  to separate author list from addresses and between each address.
%  The correspondence between each author and his/her address
%  can be indicated with a superscript in italics, 
%  which is easily obtained with \supit{}.

\author{Zachary DeStefano\supit{a}, Jack Yao\supit{a}, Nadine Abi-Jaoudeh\supit{a}, Ming Li\supit{a}
\skiplinehalf
\supit{a}National Institutes of Health, Clinical Center\\Bethesda, MD 20814
}

%>>>> Further information about the authors, other than their 
%  institution and addresses, should be included as a footnote, 
%  which is facilitated by the \authorinfo{} command.

\authorinfo{Author Emails: zdestefa@uci.edu}
%%>>>> when using amstex, you need to use @@ instead of @
 

%%%%%%%%%%%%%%%%%%%%%%%%%%%%%%%%%%%%%%%%%%%%%%%%%%%%%%%%%%%%% 
%>>>> uncomment following for page numbers
% \pagestyle{plain}    
%>>>> uncomment following to start page numbering at 301 
%\setcounter{page}{301} 
 
  \begin{document} 
  \maketitle 

%%%%%%%%%%%%%%%%%%%%%%%%%%%%%%%%%%%%%%%%%%%%%%%%%%%%%%%%%%%%% 
\begin{abstract}
Monitoring temperature during a cone-beam CT (CBCT) guided ablation procedure is important for prevention of over-treatment and under-treatment. In order to accomplish ideal temperature monitoring, a thermometry map must be generated.  Li et al explored this possibility using CBCT scans of a pig shoulder undergoing ablation. We are extending this work by using CBCT scans of real patients. Because we have real patients, we register the scans using NiftyReg **INSERT REF** before comparing them. Due to the noise in the image and remaining registration errors, we also employ a more robust change metric than simple image difference. \cite{Lamport94} 
\end{abstract}

%>>>> Include a list of keywords after the abstract 

\keywords{Cone-beam CT, CBCT, Thermometry, Radiofrequency Ablation, Time Delay Analysis}

\section{Introduction}
\label{sec:intro}  % \label{} allows reference to this section

Li et al explored the possibility of generating a thermal map from CBCT scans taken at different time points. We decided to extend this work and furthur explore thermometry map generation. We applied the methods to real patients. Because we have real patients, there is movement and deformation of the organs between CBCT scans. We thus decided to keep the baseline scan static and register the subsequent scans to the baseline one using **NIFTY**. After registration, there is still noise, needle artifacts, and small registration errors. We thus needed a change metric that was robust to these issues. We decided to compute the RMSE of the window around a given pixel. We treat the windows as a signal and find the ideal spatial delay between them and use the subsequent RMSE as the amount of change that occurred. 

The change metric we employ is computationally intensive. We thus only compute it around the Region of Interest (ROI) where the ablation occured. Because the needle has distinct Hounsfeld Unit values from tissue, we were able to isolate the needle and find the tip of it. The region around the tip was then our ROI. 

We used a data set of CBCT scans taken during 13 ablation procedures performed between September 2013 and June 2015. For each ablation procedure, there were between 1 and 4 ablations done. With each ablation, a baseline scan was taken followed by 1-4 comparison scans at different time points. Each of these time points had the ablation needle at different temperatures. We generated a thermal map at each of these time points in order to estimate the amount of tissue affected. 

\section{Related Work}

CT Thermometry has been explored since the 1970's with some success **CT OVERVIEW PAPER**. The attempts mentioned only focused on taking individual CT scan values at different points and doing a linear regression. Being that the CT scans have quite a lot of noise and artifacts, we decided to attempt to use a metric that accounted for it. Additionally, those attempts did not see what a full thermal map would look like.

Image Registration between 3D Medical Scans has been explored **NIFTY REFERENCES**. There is now an open source tool to accomplish this. We incorporated that tool in our work.  

A more robust change metric was proposed in **NADINE'S PAPER**. While this change metric would be good for classification, the formula makes it less than ideal for regression. We thus needed to figure out a change metric adapted for regression.

A change metric used in the pipeline is the difference between the convolutional average filtered image. Convolution is a common technique used in image processing. The averaging filter is a common method to reduce noise. We thus use it to give us denoised difference information. Unfortunately, it does not account for potential registration errors, so we needed something that adjusted to it. 

Our change metric ends up being RMSE between windows around a given pixel, but with an offset. Time delay analysis is a common field of study in signal processing. It is used to find the delay between signals. This is done by finding the cross correlation with different delays and seeing where it's maximized. The offset we use ends up being the same value. 

\section{Pipeline}

Here is a summary of the pipeline performed for each comparison scan to generate a thermal map:
\begin{enumerate}
\item Register the comparison scan to the baseline scan
\item Filter both images and calculate the difference
\item Use Connected Component and PCA to detect the Needles
\item Use the result to find the Region of Interest (ROI)
\item Calculate the Sliding Window RMSE value for the ROI
\item Correlate values from (5) in the temperature zones with the temperature data
\item Use regression from (6) to generate thermal map for entire ROI
\end{enumerate}


\subsection{Image Registration} 

Each comparison scan was registered to the baseline scan. Affine **INSERT REF** and Deformable Registration **INSERT REF** were performed using NiftyReg. Figure \ref{beforeAndAfterReg} shows an example of image slices before and after registration. 

\begin{figure} 
\centering 
\begin{subfigure}[t]{0.45\textwidth} 
\includegraphics[width=\textwidth]{unregisteredSlice.png}
\caption{Before Registration} 
\label{unregSlice} 
\end{subfigure} 
\begin{subfigure}[t]{0.45\textwidth} 
\includegraphics[width=\textwidth]{registeredSlice.png} 
\caption{After Registration} \label{regSlice} 
\end{subfigure} 
\caption{Baseline (green channel) and Comparison (red channel) Image superimposed in one image before and after registration}
\label{beforeAndAfterReg} 

\end{figure}

\subsection{Needle Detection}

The HU values for the needle and thermocouple are significantly higher than the values for normal tissues. Thus after thresholding the values in the CBCT scan high enough, the points remaining are the ones in the needle and thermocouple. Doing connected component analysis **INSERT REF** let us find the clusters which contain the needles and thermocouples. For each components, PCA was done to find the needle vector and endpoints. 

Here was the procedure done to get the ROIs:
\begin{enumerate}
\item Convert a voxel to a 3D point in the set if the HU value is over 1200
\item Run connected component analysis
\item Isolate the connected components with the most points
\item For each component:
\item Run PCA and find most significant vector
\item Find each points coordinate in the new vector space
\item The points with the greatest and least value will be the endpoints
\end{enumerate}

\begin{figure} 
\centering 
\begin{subfigure}[t]{0.45\textwidth} 
\includegraphics[width=\textwidth]{needleDetection3D_1.png}
\end{subfigure} 
\begin{subfigure}[t]{0.45\textwidth} 
\includegraphics[width=\textwidth]{needleDetection3D_2.png} 
\end{subfigure} 
\caption{Needles and Thermocouple (blue) and their endpoints (red) detected using PCA}
\label{needleDetection} 
\end{figure}

\subsection{Change Detection}

These images have a low signal to noise ratio as well as beam hardening artifacts. Li et al used the Wronskian Change Detector to try to account for this. While this change detector works well for classifying which pixels changed, it is not ideally suited for regression where the amount of change is important. The Wronskian detector uses a window around a pixel. We decided to use the window method also but with a different change metric. 

We ended up using two metrics. The first one is more efficient but less robust. The second more is less efficient to compute but much more robust. The first one was used to get an idea of what the thermal map will look like. The second one was used to actually generate the thermal map. 

There are various image filters that are meant to reduce noise. We decided to incorporate an averaging filter. We apply the filter to both images and then find the difference of the filtered image. This is equivalent to taking a window around a pixel and finding the mean difference in the window. This was the first method. 

While the first method reduces the effects of noise, there are still beam hardening artifacts and registration errors that are not accounted for. We thus incorporated a second method where we took the windows and minimized the RMSE over every offset of the windows. 

The second method is equivalent to treating the baseline and comparison windows as signals and finding the ideal spatial delay between them **INSERT REF**. 

Here is the second method procedure:
For each pixel $(i,j)$ in the result image:
\begin{enumerate}
\item Let $w$ be half-width of the window around the pixel. 
\item Let $n=2w+1$. This means the window is size $n$ by $n$. 
\item Let $U$ be window around $(i,j)$ in baseline image
\item Let $U'$ be $U$ repeated in the following way:
\[
\forall(k,l,i \in Z)\, \, \, U'_{k+i \cdot n,l+i \cdot n} = U_{k,l}
\]
\item Let $V$ be window around $(i,j)$ in the comparison image
\item Calculate the following:
\[
\min_{r,c} \sum_{l=0}^{n-1} \sum_{k=0}^{n-1} {(U'_{k+r,l+c}-V_{k,l})^2}
\]

\end{enumerate}

\begin{figure} 
\centering 
\includegraphics[width=\textwidth]{changeDetectionPanel2.png} 
\caption{Results of different change metrics in the ROI. Raw subtraction (left), Filtered Difference (center), and Sliding Window RMSE (right)} 
\end{figure}

\subsection{Regression}

After generating the sliding window image, we find the coordinates where the measured temperature zones are located. We correlate the sliding window values with the measured temperature at those zones using linear regression. We then take the regression equation found and compute its output across the sliding window image. This gives us the thermal map.  

We then took the sliding window RMSE value and calibrated it using the measured temperature change at the needle through a regression model. We used the model to calculate a thermal map in the ROI. This thermal map was used to obtain an approximate mean temperature around the needles and thermocouples.

\begin{figure} 
\centering 
\includegraphics[width=4in]{slidingDiffRegression.png} 
\caption{Regression Curve Calculated} 
\end{figure}

\begin{figure} 
\centering 
\includegraphics[width=4in]{slidingDiffThermalMap.png} 
\caption{Thermal Map generated from Regression curve} 
\end{figure}


\section{Results}

\subsection{Finding Error Rates}

We decided to take the thermal maps generated from regression and see the average temperature in the zones that we were testing. We then did an RMSE of those temperatures with the measured temperatures. We dividied this RMSE by the temperature range in order to obtain an error rate. Here are the results:

**INSERT RESULT TABLE**

\subsection{Approximating Ablation Zone Area}

For the ablation zones, we made a graph of radius of region versus average temperature in that region in the thermal map. This graph can one day be used to approximate the radius of an ablation zone. Here it is for the various patients in our study. 

\begin{figure} 
\centering 
\includegraphics[width=3in]{meanTempVsRadiusFull.png} 
\caption{Shows the graph of radius vs mean temperature} 
\end{figure}


\subsection{3D Visualization of Ablation Zone}

By taking the 3D Thermal Map, thresholding the values, and then using a surface generation algorithm **REFERENCE ITK-SNAP HERE**, we were able to create a 3D diagram of the ablation zone. We overlayed the 3D diagram of the needle and we were able to obtain a convincing visual. 

\begin{figure} 
\centering 
\includegraphics[width=3in]{NeedleAndAblationZone.png} 
\caption{Shows a 3D Visualization of the Needle and Ablation Zone} 
\end{figure}

\section{Conclusions and Future Work}

As can be observed, for some of the patients, our temperatures after regression were quite accurate while for others, the error was higher. The higher error was the result of a high residual value in the linear regression. This was likely caused by image noise, registration error, or user error when selecting the ROI. 

For the patients where the error was low, the methods we employed have the potential to provide useful thermal maps during ablation procedures. These thermal maps can then be used to approximate the size of the ablation zone. Additionally, there is the potential to generate a 3D visual representation of the ablation zone. 

The next step is to automate ROI selection using the needle properties. Additionally, we hope to use a data set where multiple imaging modalities were employed. That way we can test our thermal map against one generated by a modality that is known to generate accurate thermal maps. 



\bibliography{report}   %>>>> bibliography data in report.bib
\bibliographystyle{spiebib}   %>>>> makes bibtex use spiebib.bst

\end{document} 
